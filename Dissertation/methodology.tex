\title{Methodology}
 This section will cover all the methodologies that were used in the development of this project, as
 well as the rational behind them. This section also covers how we handled meetings, project management and how we worked around not being able to meet in person during the first semester and working remotely. We did this by utilizing Methodologies, such as Agile, Scrum and Test Driven Development. It also covers the tools we used to manage our code, meetings and schedules.
 
\section{Initial Plan}
\subsection{Brainstorming}
The first step in our plan was a brain storming session were we discussed different ideas for the project and factored in the different technologies that we were interested in learning. We decided on a Web Application that would aid with Table Top Roleplaying Game's with a focus on Dungeons and Dragons, as organizing a session during travel restrictions proved to be difficult task to manage. We quickly realized there was an opportunity to familiarize ourselves with these new technologies including a NoSQL database in firebase and Microsoft Azure. Afterwards we drew up a basic storyboard of what we would ideally like the website to look like and what features we would wanted to implement into it's design., as well as how difficult a task these would be.

\subsection{Research}
We used Qualitative analysis to research our idea to see if there were any samples of this already on the market and found there was nothing that precisely matched our concept. This helped us to narrow down the scope of our project and focus our efforts on specific areas such as the social media aspect of the project, as there were websites that already had detailed 'Character Creators' such as 'Orc Pub', but none where you could share ideas and information between users , while guiding you through some of the more technical aspects of the game.

\subsubsection{Front End}
After researching Angular, REACT and Ionic we decided on using Angular as we found a single page application would suit the project's needs as we wanted to try and compact all relevant information onto a small screen  to allow the Web App be easily navigate and comprehended, with information relevant to the user. As well as this, the many libraries that Angular includes that makes it flexible when connecting with the other components of the project such as AngularFirestore which contains plugins to aid with Angular to Firebase development and Visual Studio Code having angular specific extensions.

\subsubsection{Cloud Services}
After deciding on what we wanted to do, we focused our research on what platforms we wanted to use for the application.  We compared Cloud Platforms such as Microsoft's Azure and Amazon Web Services, to see what the individual strengths and weaknesses of these platforms were. We found to be equally suitable for the project. We chose Azure as we had used AWS before and were excited to try a different platform.

\subsubsection{Databases}
Next we researched databases such as MongoDB and Google Firebase and decided to use Firebase as we had not previously worked on a NoSQL database and the amount we would learn for designing and implementing it would be vast and invaluable. Another advantage of Firebase was Angular had libraries such as AngularFirestore that helped connect our Front end of the project with the database.

 
 \section{Meetings}
 \subsection{Project Management Methodologies}
 \subsubsection{Agile}
  Agile is a Iterative approach to Software Development that aims to shorten the development life cycle, by increasing productivity and to regularly reassess the project as meeting in person was not a viable option. We immediately saw that this approach was far more suited to the project then the traditional Waterfall Methodology, as the flexibility Agile provides to adapt to a problem was perfectly suited to our development with several new technologies. Agile sets short term goals that are set be completed in a finite period of time called sprints. This included regular iterative testing throughout the project during theses sprints.
  
 \subsubsection{Scrum}
 We decided to use SCRUM as our Agile Methodology of choice. We believed this suited our project as it emphasizes short term goal orientated development as well as a respectful amount of independent work. This suited our remote working schedule, while allowing us to regularly discuss our progress and reevaluate our progress and planning in our weekly Scrum meetings. It is also well suited for receiving feedback and quickly implementing it into our plan as we would only have to readjust our short term goals. This was ideal when working with new technologies as we weren't sure what aspects might be easier or tougher then our original expectations during our  initial research and planning.
 
 \subsubsection{Test Driven Development}
 We decided to use Test Driven Development as we knew there would be a learning curve with aspects of the project such as building a fully functioning database with a new language. It was vital to test the code regularly as we went along. Especially due to that fact that if we made any mistakes it was vital to figure it out early and fix the error before other features of the project were to become reliant on something that would not be suited to what we needed it to do.
 
 \subsection{Discussion Platforms}
 \subsubsection{Microsoft Teams}
 We had weekly meetings with our supervisor Daniel Cregg remotely on Microsoft Teams, where we would discuss our progress and send a digital report of what we had achieved and our goals for the next week. This was our main Scrum meeting where we would set our objectives for the upcoming sprint and discuss what we difficulties we found and were able to adjust our plan. Here we would received crucial feedback that would greatly aid our development and speed up our decision making with the insight we were given, especially with regards on how to document the project regularly throughout the development.
 
 \subsubsection{Discord}
 After our meetings with our supervisor we used Discord, a voice chat and messaging app to meet and discuss our project further in detail and plan our workflow and schedules for the week, as well as sharing some ideas and documentation we found.
 
 \section{Development Tools}
 \subsection{IDE}
 Our Integrated Development Environment of choice was Visual Studio code as it is a relatively lightweight compared to other common IDEs such as Eclipse. Visual Studio was created by Microsoft and as such is easily integrated with Azure which is one of the reason we choose it. One of the main reasons we chose Visual Studio Code was because of its vast amount of extensions for various languages and quality of life features that aid in the development. The fact that the community can add small extensions for certain preferences  is a big advantage.
 
 \subsection{Console}
 \subsubsection{Cmder}
 Cmder is a software package that allows the user to use Unix commands on windows as well as customization including history of previous commands used which proved to be a handy tool when regularly committing to Github.
 
 \subsection{Source Control}
 \subsubsection{Github and Git}
 Github is a hosting site used by coders which was first launched in 2008, that utilizes Git for it's version controls. It allows for easier collaboration on a single project as well as creating a backup of the project online. It allows the user to revert to a previous version or to create branches to work on certain features of the project before merging it back into the main branch of the project.
 
 \subsection{Testing}
 \subsubsection{Manual Testing}
 We did regular manual testing to make sure all methods worked throughout the development, especially before committing to our repository. This was key early on to make sure the foundations of the project and the connections between the front end, back end and database worked as intended, as they would be needed to call on later on in development.
 
 \subsection{Documentation}
 \subsubsection{Overleaf}
 We used overleaf to write the dissertation as it allowed to professionally present and customize our layout of a Latex document. Much like Github, the big advantage of overleaf is that it is stored online and compiles, which allows us to work simultaneously on the documentation in real time.
 
 \subsubsection{Journal}
 As we progressed through our project we recorded who worked on what feature in the Journal.md file on the github repository, as well as some insights into our rational for design decisions and bugs encountered along the way.